\documentclass[12pt, reqno]{amsart}
\usepackage[utf8]{inputenc}
\usepackage{amsmath, amsfonts, amssymb, amsthm, setspace, hyperref}
\numberwithin{equation}{section}
\parindent 0mm
\onehalfspacing
\addtolength{\textwidth}{2 truecm}
\setlength{\hoffset}{-1 truecm}

\newcommand{\e}{\epsilon}
\newcommand{\R}{\mathbb{R}}
\newcommand{\N}{\mathbb{N}}
\newcommand{\Z}{\mathbb{Z}}
\newcommand{\w}{\omega}
\let\oldemptyset\emptyset
\let\emptyset\varnothing
\let\phi\varphi

\newcommand{\intersect}{\bigcap}
\newcommand{\union}{\bigcup}
\newcommand{\disjointunion}{\coprod} 
\newcommand{\powerset}{\mathcal{P}}
\newcommand{\dist}{\sim}
\renewcommand{\P}{P}
\newcommand{\E}{\text{E}}
\newcommand{\Var}{\text{Var}}
\newcommand{\given}{\mid}
\newcommand{\mean}{\bar}

\begin{document}

\title{Change of Variables in Integration}
\author{Austin David Brown}
\maketitle

TODO prove all of these

TODO add examples

\section{Change of Variables in Integration}

THE BIG IDEA IS GOING FROM THE DOMAIN TO CODOMAIN OF SOME FUNCTION!

The main idea behind change of variables is changing the integration from the domain to the codomain of some function.

\textbf{COV 1}

See Linear Algebra Done Right for some insight.

Let $f : \R^n \to [0, +\infty]$ be measurable.
Let $T \in GL_n(\R).$
Since $T$ is linear, $vol(T(\R^n)) = |det(T) | vol(\R^n)$. Hence,
\[
\int_{T(\R^n)} f d\mu
= | det(T) | \int_{R^n} f \circ T d\mu.
\]

\textbf{COV 1 for differentiation}

Let $f : \R^n \to [0, +\infty]$ be measurable.

Let $\phi : U \to \R^n$ be a diffeomorphism.

Since the differential $\phi(x) + D\phi(x) : U \to \R^n$ is a linear approximation to $\phi$ near $x$, shifting by $\phi(x)$ does not change volume, and since $D\phi(x)$ is linear,
\[
\int_{\phi(U)} f(y) dy
= \int_{U} f \circ \phi (x) |det( D\phi(x) ) | dx.
\]

\textbf{COV 2}

\textbf{Lebesgue integral version}:

This is the pushforward change of variable. The idea is that the way that this pushforward measure is defined, there is no change in volume.

Let $f : X \to [0, +\infty]$ be a measurable map. Let $g : X \to Y$ be a measureable map. If $\mu$ is a measure on $X$, then $\mu g^{-1}$ is a pushforward measure or image measure on the sigma algebra of $Y$.
Then
\[
\int_{g(X)} f(y) d \mu(g^{-1}(y))
= \int_X f(g(x)) d \mu(g g^{-1} (x)).
= \int_X f(g(x)) d \mu(x).
\]

\textbf{Example}

The main change of variables for random variables is this.
If $f = id$, and $X : \Omega \to \R$ is a $RV$, then
\[
E(X) = \int_{\Omega} X(\omega) dP(\omega)
= \int_{X(\Omega)} x dPX^{-1}(x).
\]

By definition,
\[
\int_{X(\Omega)} x dPX^{-1}(x)
= \int_{X(\Omega)} x dP_X(x)
= \int_{X(\Omega)} x dF_X(x).
\]

\textbf{COV 3}

\textbf{Riemann-Stieltjes integral version}:

Let $f : $[a, b]$ \to \R$ be Riemann integrable.
Let $g : [a, b] \to \R$ be continuous and $g'$ is Riemann integrable.
Then
\[
\int_{[a, b]} f(x) d(g(x))
= \int_{[a, b]} f(x) g'(x) dx
\]

\textbf{Lebesgue integral version:}

Let $g$ be absolutely contiuous and it will hold.

\textbf{Examples:}

This one is useful when $g = F_X$ is the distribution function of a random variable.

\end{document}
